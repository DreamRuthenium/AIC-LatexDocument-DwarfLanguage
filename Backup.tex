\documentclass{article}
\usepackage{graphicx} % Required for inserting images
\usepackage{fontspec} % For English display
\usepackage{xeCJK} % For Chinese & Japanese display
\usepackage{multicol}
\usepackage{graphicx}
\usepackage{tikz}
\usepackage{xcolor}
\usepackage{xparse}
\usepackage{subcaption}
\usepackage{tabularx}
% BE CAREFULE: IF YOU WANT TO ADD CHINSES, DO NOT EDIT THIS FILE
% GO TO Chinese.tex

\setlength{\parindent}{0pt}

%%%%%%%%%%%%%%%%%%%%%%%%%%%%%%%%%%%%%%%%%%%%%%%%%%%%%%%%%%%%%%%
% File structure
%
% Root
%root/
%├── Figs/                            You should put all figures here
%│
%├── Alice.ttf                        ttf file for the dwarf language
%│
%├── Alice.kbitx                      design file for dwarf language
%│
%├── BitsNPicas.jar                   Program to edit Alice.kbits
%│
%├── Japanese.tex                     Japanese tex file
%└── Chinese.tex                      Chinese tex file
%%%%%%%%%%%%%%%%%%%%%%%%%%%%%%%%%%%%%%%%%%%%%%%%%%%%%%%%%%%%%%%

\definecolor{mygreen}{RGB}{30,90,55}
\definecolor{myorange}{RGB}{234,102,44}
\definecolor{myred}{RGB}{170,0,41}

%%%%%%%%%%%%%%%%%%%%%%%%%%%%%%%%%%%%%%%%%%%%%%%%%%%%%%%%%%%%%%%
% Set some fonts for display
% I chosen them based on my taste, feel free change them if needed
% These are supported fonts:
% https://www.overleaf.com/learn/latex/Questions/Which_OTF_or_TTF_fonts_are_supported_via_fontspec%3F#!CJK
%%%%%%%%%%%%%%%%%%%%%%%%%%%%%%%%%%%%%%%%%%%%%%%%%%%%%%%%%%%%%%%
\setmainfont{x12y16pxMaruMonica.ttf} % Eng display
\newCJKfontfamily\zhfont{Noto Sans CJK HK} % 中文
\newCJKfontfamily\jpfont{x12y16pxMaruMonica.ttf} % 日文
\newfontfamily\PUAfont{Alice.ttf}[
  Range = {U+0100-U+017F}
]

%%%%%%%%%%%%%%%%%%%%%%%%%%%%%%%%%%%%%%%%%%%%%%%%%%%%%%%%%%%%%%%
% Paper length & width
%%%%%%%%%%%%%%%%%%%%%%%%%%%%%%%%%%%%%%%%%%%%%%%%%%%%%%%%%%%%%%%
\usepackage[
  a4paper,          
  left=1cm,
  right=1cm,
  top=1.5cm,
  bottom=1.5cm
]{geometry}

\begin{document}
\textbf{\jpfont{すべての英語コンテンツには日本語訳が付いています。}}
\section{Introduction / \normalfont{\jpfont{前置き}}}
\subsection{Credibility / \normalfont{\jpfont{信頼性}}}
We use colors to represent the credibility of the translation:\\
{\color{mygreen}Green}: The result is 100\% correct.(Unless their meaning changed in further versions).\\
{\color{myorange}Orange}: We are confident in the result, but we can't promise they are correct.\\
{\color{myred}Red}: We are not confident in the result, in other words, they are basically random guessing.\\\\
Most translations are in {\color{myorange}orange state}, because we will only set translation to {\color{mygreen}green state} when we have very strong evidence to prove the credibility(which will be introduced later).\\\\
Also, sometimes it is too difficult to translate some sentences, since most of their words appeared only once. Under that circumstance, we would use "({\color{myred}[We are unable to translate this]})({\color{myred}\jpfont{[合理的な翻訳は得られない]}})" to mark them. Luckily, most sentences are translatable.\\\\
\jpfont{翻訳の信頼性を表すために色を使用しています:}\\
\jpfont{{\color{mygreen}緑色: }結果は100\%正確だ。}\\
\jpfont{{\color{myorange}オレンジ: }結果には自信がありますが、それが正しいとは保証できません。}\\
\jpfont{{\color{myred}赤色: }結果には自信がありません。言い換えれば、推測だけです。}\\\\
\jpfont{ほとんどの翻訳は{\color{myorange}オレンジ状態}です。これは、信頼性を証明する非常に強力な証拠がある場合にのみ、翻訳を{\color{mygreen}緑色状態}に設定するためです(証拠は後で紹介されます)。}\\\\
\jpfont{また、文を構成するシンボルに一度しか登場しないシンボルの数が多いすぎる場合では、翻訳ができない場合もあります。その時、"({\color{myred}[We are unable to translate this]})({\color{myred}\jpfont{[合理的な翻訳は得られない]}})"を使ってマークします。幸いなことに、翻訳できないの比例は少ないです。}

\subsection{References / \normalfont{\jpfont{参考資料}}}
In addition to ordinary materials like stream video, we have also got some very powerful hint by unpacking the game. \\\\
As shown in the figure (a) , we have found this picture from one released game version. Many symbols occurred in this manuscript are also used in the preview version, so this picture is basically a "Trust Anchor" for our translation.\\
As shown in figure (b), we are able to call some dwarf symbols and their pronunciation in 0.28i version. We have developed a hypothesis based on it: Most dwarf words pronounce the leading syllable of it's Japanese translation. For example, {\PUAfont \symbol{"0120}}, whose meaning is "You", reads as "Ki", and the pronunciation of "You" in Japanese is "Ki Mi".\\\\
\jpfont{配信ビデオで得た資料に加えて、ゲームをデシリアライゼーションすることによって、役に立つ情報も得ました。}\\\\
\jpfont{図(a)に示すように、あるAICバージョンからこの原稿を発見しました。この原稿に登場する多くの記号はセリフでも使用されているため、この原稿は基本的に私たちの翻訳の「信頼アンカー」と言えます。}\\
\jpfont{図(b)に示すように、バージョン0.28iで一部の「dwarf語」シンボルとその日本語の発音を呼び出せました。私たちはこの情報に基づいて仮説を立てました。ほとんどの「dwarf語」シンボルは日本語訳の先頭の音節を発音します。例えば、{\PUAfont \symbol{"0120}}の意味は”きみ”、そしてその発音は”き”。}

\begin{figure}[htbp]
\begin{subfigure}{0.45\textwidth}
    \centering
  \includegraphics[height=6cm]{Figs/Others/character_dwarf.png}
  \caption{Hashino's manuscript / \jpfont{はーちゃんの原稿}}
\end{subfigure}
\begin{subfigure}{0.45\textwidth}
  \centering
  \includegraphics[height=6cm]{Figs/Others/loaded_4_3.png}
  \caption{Japanese pronunciation of some symbols / \jpfont{いくつかのシンボルの発音}}
\end{subfigure}

\end{figure}


\newpage
%%%%%%%%%%%%%%%%%%%%%%%%%%%%%%%%%%%%%%%%%%%%%%%%%%%%%%%%%%%%%%%
% In this section, you should write:
% 1. A table showing all discovered symbols
%%%%%%%%%%%%%%%%%%%%%%%%%%%%%%%%%%%%%%%%%%%%%%%%%%%%%%%%%%%%%%%
\section{All discovered dwarf symbols / \normalfont{\jpfont{発見したシンボル}}}
%%%%%%%%%%%%%%%%%%%%%%%%%%%%%%%%%%%%%%%%%%%%%%%%%%%%%%%%%%%%%%%
% Table showing all alice-language words and their translations
%%%%%%%%%%%%%%%%%%%%%%%%%%%%%%%%%%%%%%%%%%%%%%%%%%%%%%%%%%%%%%%
\begin{table}[htbp]
  \Large
  \begin{minipage}[t]{0.15\textwidth}
    \vspace{0pt}
    \centering
    \begin{tabular}{ll}
        \hline
        Symbols & code \\
        \hline
        {\PUAfont \symbol{"0100}} & 0100\\
        {\PUAfont \symbol{"0101}} & 0101\\
        {\PUAfont \symbol{"0102}} & 0102\\
        {\PUAfont \symbol{"0103}} & 0103\\
        {\PUAfont \symbol{"0104}} & 0104\\
        {\PUAfont \symbol{"0105}} & 0105\\
        {\PUAfont \symbol{"0148}} & 0148\\
        {\PUAfont \symbol{"0106}} & 0106\\
        {\PUAfont \symbol{"0107}} & 0107\\
        {\PUAfont \symbol{"0108}} & 0108\\
        {\PUAfont \symbol{"0109}} & 0109\\
        {\PUAfont \symbol{"010A}} & 010A\\
        {\PUAfont \symbol{"010B}} & 010B\\
        {\PUAfont \symbol{"010C}} & 010C\\
        {\PUAfont \symbol{"010D}} & 010D\\
        {\PUAfont \symbol{"010E}} & 010E\\
        {\PUAfont \symbol{"010F}} & 010F\\
        {\PUAfont \symbol{"0153}} & 0153\\
        {\PUAfont \symbol{"0110}} & 0110\\
        {\PUAfont \symbol{"0111}} & 0111\\
        {\PUAfont \symbol{"0112}} & 0112\\
        {\PUAfont \symbol{"0113}} & 0113\\
    \end{tabular}
  \end{minipage}
  \hfill
  \begin{minipage}[t]{0.15\textwidth}
    \vspace{0pt}
    \centering
    \begin{tabular}{ll}
        \hline
        Symbols & code \\
        \hline
        {\PUAfont \symbol{"0115}} & 0115\\
        {\PUAfont \symbol{"0116}} & 0116\\
        {\PUAfont \symbol{"014B}} & 014B\\
        {\PUAfont \symbol{"014C}} & 014C\\
        {\PUAfont \symbol{"0117}} & 0117\\
        {\PUAfont \symbol{"0118}} & 0118\\
        {\PUAfont \symbol{"0119}} & 0119\\
        {\PUAfont \symbol{"011A}} & 011A\\
        {\PUAfont \symbol{"011B}} & 011B\\
        {\PUAfont \symbol{"011C}} & 011C\\
        {\PUAfont \symbol{"011D}} & 011D\\
        {\PUAfont \symbol{"0150}} & 0150\\
        {\PUAfont \symbol{"011E}} & 011E\\
        {\PUAfont \symbol{"011F}} & 011F\\
        {\PUAfont \symbol{"0120}} & 0120\\
        {\PUAfont \symbol{"0121}} & 0121\\
        {\PUAfont \symbol{"014E}} & 014E\\
        {\PUAfont \symbol{"0122}} & 0122\\
        {\PUAfont \symbol{"0123}} & 0123\\
        {\PUAfont \symbol{"0124}} & 0124\\
        {\PUAfont \symbol{"0125}} & 0125\\
        {\PUAfont \symbol{"014A}} & 014A\\
        
        
        
    \end{tabular}
  \end{minipage}
  \hfill
  \begin{minipage}[t]{0.15\textwidth}
    \vspace{0pt}
    \centering
    \begin{tabular}{ll}
        \hline
        Symbols & code \\
        \hline
        {\PUAfont \symbol{"0126}} & 0126\\
        {\PUAfont \symbol{"0127}} & 0127\\
        {\PUAfont \symbol{"0128}} & 0128\\
        {\PUAfont \symbol{"014D}} & 014D\\
        {\PUAfont \symbol{"014F}} & 014F\\
        {\PUAfont \symbol{"0129}} & 0129\\
        {\PUAfont \symbol{"012A}} & 012A\\
        {\PUAfont \symbol{"012B}} & 012B\\
        {\PUAfont \symbol{"012C}} & 012C\\
        {\PUAfont \symbol{"0149}} & 0149\\
        {\PUAfont \symbol{"012D}} & 012D\\
        {\PUAfont \symbol{"012E}} & 012E\\
        {\PUAfont \symbol{"012F}} & 012F\\
        {\PUAfont \symbol{"0130}} & 0130\\
        {\PUAfont \symbol{"0131}} & 0131\\
        {\PUAfont \symbol{"0132}} & 0132\\
        {\PUAfont \symbol{"0133}} & 0133\\
        {\PUAfont \symbol{"0134}} & 0134\\
        {\PUAfont \symbol{"0135}} & 0135\\
        {\PUAfont \symbol{"0136}} & 0136\\
        {\PUAfont \symbol{"0147}} & 0147\\
        {\PUAfont \symbol{"0137}} & 0137\\
        
        
        
        
    \end{tabular}
  \end{minipage}
\hfill
  \begin{minipage}[t]{0.15\textwidth}
    \vspace{0pt}
    \centering
    \begin{tabular}{ll}
        \hline
        Symbols & code \\
        \hline
        {\PUAfont \symbol{"0138}} & 0138\\
        {\PUAfont \symbol{"0139}} & 0139\\
        {\PUAfont \symbol{"013A}} & 013A\\
        {\PUAfont \symbol{"013B}} & 013B\\
        {\PUAfont \symbol{"013C}} & 013C\\
        {\PUAfont \symbol{"013D}} & 013D\\
        {\PUAfont \symbol{"013E}} & 013E\\
        {\PUAfont \symbol{"013F}} & 013F\\
        {\PUAfont \symbol{"0140}} & 0140\\
        {\PUAfont \symbol{"0141}} & 0141\\
        {\PUAfont \symbol{"0142}} & 0142\\
        {\PUAfont \symbol{"0154}} & 0154\\
        {\PUAfont \symbol{"0143}} & 0143\\
        {\PUAfont \symbol{"0144}} & 0144\\
        {\PUAfont \symbol{"0145}} & 0145\\
        {\PUAfont \symbol{"0146}} & 0146\\
        {\PUAfont \symbol{"0151}} & 0151\\
        {\PUAfont \symbol{"0152}} & 0152\\
        {\PUAfont \symbol{"017E}} & 017E\\
        {\PUAfont \symbol{"017F}} & 017F\\
        
    \end{tabular}
  \end{minipage}
\end{table}


We drew all symbols as 16x16 pixel arts and encoded them as Latin-Extended-A Unicode, so they can be easily input in the form of Unicode.\\\\
Some appeared dwarf symbols are in Italics. Due to the lack of text messages, we cannot accurately know if a character is in Italics or not. Currently, we only find Italics in names.\\
{\PUAfont \symbol{"0125}}(0125) and {\PUAfont \symbol{"014A}}(014A) are very close, but there are really two types of character that appeared in the live video. We believe this is a display mistake, but these two characters are still distinguished in case we are wrong.\\\\
\jpfont{すべてのシンボルを 16x16 pixel art として模写し、Latin-Extended-A Unicode にエンコードしたので、Unicode 形式で簡単に入力できます。}\\\\
\jpfont{セリフに出るシンボルの一部はイタリック体で表示されています。テキスト情報は少ないため、文字がイタリック体かどうかを100\%正確に判断できません。今のところ、名前だけにイタリック体の使用が多を発見しました。}\\
\jpfont{{\PUAfont \symbol{"0125}}(0125) と {\PUAfont \symbol{"014A}}(014A) はかなり似ているけど、配信ビデオにはこれらは確かに二種類のシンボルです。これはディスプレイミスだと考えていますが、間違っている場合に備えて、これら2つの文字は区別されています。}
\newpage



%%%%%%%%%%%%%%%%%%%%%%%%%%%%%%%%%%%%%%%%%%%%%%%%%%%%%%%%%%%%%%%
% In this section, you should write:
% 1. Translation of known words
% 2. Symbols that are still untranslatable
%%%%%%%%%%%%%%%%%%%%%%%%%%%%%%%%%%%%%%%%%%%%%%%%%%%%%%%%%%%%%%%
\section{Translation of symbols and words / \normalfont{\jpfont{単語とシンボルの翻訳}}}
\subsection{Pronunciation / \jpfont{発音}}
These are symbols that we know how to pronounce.\\
\jpfont{これらは発音方法を知っているシンボルです。}
\begin{table}[htbp]
  \Large
  \begin{minipage}[t]{0.15\textwidth}
    \vspace{0pt}
    \centering
    \begin{tabular}{ll}
        \hline
        Symbols & Pronunciation \\
        \hline
        {\PUAfont \symbol{"011B}} & a/\jpfont{あ}\\
        {\PUAfont \symbol{"0149}} & i/\jpfont{い}\\
        {\PUAfont \symbol{"012C}} & u/\jpfont{う}\\
        {\PUAfont \symbol{"0121}} & e/\jpfont{え}\\
        {\PUAfont \symbol{"014C}} & o/\jpfont{お}\\
    \end{tabular}
  \end{minipage}
  \hspace{5em}
  \begin{minipage}[t]{0.15\textwidth}
    \vspace{0pt}
    \centering
    \begin{tabular}{ll}
        \hline
        Symbols & Pronunciation \\
        \hline
        {\PUAfont \symbol{"014B}} & ka/\jpfont{か}\\
        {\PUAfont \symbol{"0120}} & ki/\jpfont{き}\\
        {\PUAfont \symbol{"010C}} & ku/\jpfont{く}\\
        {\PUAfont \symbol{"014D}} & ko/\jpfont{こ}\\
        {\PUAfont \symbol{"014E}} & su/\jpfont{す}\\
        {\PUAfont \symbol{"0138}} & su/\jpfont{す}\\
    \end{tabular}
  \end{minipage}
  \hspace{5em}
  \begin{minipage}[t]{0.15\textwidth}
    \vspace{0pt}
    \centering
    \begin{tabular}{ll}
        \hline
        Symbols & Pronunciation \\
        \hline
        {\PUAfont \symbol{"014F}} & so/\jpfont{そ}\\
        {\PUAfont \symbol{"0117}} & ha/\jpfont{は}\\
        {\PUAfont \symbol{"011D}} & wa/\jpfont{わ}\\
        {\PUAfont \symbol{"0152}} & bi/\jpfont{び}\\
        {\PUAfont \symbol{"010F}} & ri/\jpfont{り}\\
        
    \end{tabular}
  \end{minipage}
\end{table}

\subsection{Punctuation marks / \jpfont{句読点}}
1. {\PUAfont \symbol{"017F}}\\
{\color{mygreen}Same as usage of "!", this symbol appears at the end of most sentences except for interrogative sentences.\\
\jpfont{このシンボルは疑問文を除くほとんどの文の末尾に表示されます。"!"の使い方と同じです。}\\}
2. {\PUAfont \symbol{"0102}}\\
{\color{mygreen}This symbol appears at the \textbf{START} of interrogative sentences, and the sentence do not end with other punctuation marks.\\
\jpfont{このシンボルは疑問文の冒頭に表示されます。その文の結尾には他の句読点が無くてもいい。}\\}
3. {\PUAfont \symbol{"0104}}\\
{\color{myred}This symbol is used to modify rhetorical questions.\\
\jpfont{この記号は修辞的な疑問を修飾するために使用されます。}\\}
4. {\PUAfont \symbol{"0103}}{\PUAfont \symbol{"0104}}\\
({\color{myred}[We are unable to translate this]})\\
({\color{myred}\jpfont{[合理的な翻訳は得られない]}})

\subsection{Name / \jpfont{名前}}
1. {\PUAfont \symbol{"0101}}{\PUAfont \symbol{"0148}}{\PUAfont \symbol{"0147}}\\
{\color{mygreen}Name of Alice's Pet, since we don't know how to read it, now it is tranlsated as [Pet's name].}\\
\jpfont{{\color{mygreen}アリスのペットの名前, 私たちはこの単語の読み方は知っていないため、[ペットの名前]に書きます。}}\\
2. {\PUAfont \symbol{"011B}}{\PUAfont \symbol{"010F}}{\PUAfont \symbol{"0138}}\\
{\color{mygreen}Name of Alice.}\\
{\color{mygreen}\jpfont{アリスの名前。}}

\subsection{Exclamative particle / \jpfont{感嘆詞}}
Exclamative particles consists of one or multiple {\PUAfont \symbol{"0118}}, {\PUAfont \symbol{"0119}}, {\PUAfont \symbol{"0135}}, {\PUAfont \symbol{"012C}}, {\PUAfont \symbol{"0117}}, {\PUAfont \symbol{"017E}}, {\PUAfont \symbol{"0137}}.\\
\jpfont{感嘆詞は、1 つまたは複数の{\PUAfont \symbol{"0118}}, {\PUAfont \symbol{"0119}}, {\PUAfont \symbol{"0135}}, {\PUAfont \symbol{"012C}}, {\PUAfont \symbol{"0117}}, {\PUAfont \symbol{"017E}}, {\PUAfont \symbol{"0137}}で構成されます。}\\
Knowing these words are exclamative particles is simple, but knowing their pronounciation is quite hard. We can only guess some potential reading method based on context and plot.\\
\jpfont{これらが感嘆詞であることは簡単に分かりますが、発音を知るのは非常に困難です。文脈と筋書きに基づいて、読み方を推測することしかできません。}\\
1. {\PUAfont \symbol{"012C}}{\PUAfont \symbol{"0117}}{\PUAfont \symbol{"0117}}{\PUAfont \symbol{"0117}}\\
{\color{mygreen}Laughter. It reads as "UHahahaha!\\
「ウハハハ」に読める。\\}
2. {\PUAfont \symbol{"0119}}{\PUAfont \symbol{"010C}}{\PUAfont \symbol{"010C}}\\
{\color{mygreen}Mild laughter. It reads as "Heh, heh".\\
「っくく」に読める。\\}
3. {\PUAfont \symbol{"0119}}{\PUAfont \symbol{"0135}}{\PUAfont \symbol{"0135}}{\PUAfont \symbol{"0135}}{\PUAfont \symbol{"0135}}{\PUAfont \symbol{"0135}} / {\PUAfont \symbol{"0135}}{\PUAfont \symbol{"0119}} / {\PUAfont \symbol{"0135}}{\PUAfont \symbol{"017E}}{\PUAfont \symbol{"017E}}\\
It is very strange that {\PUAfont \symbol{"0119}} and {\PUAfont \symbol{"0135}} can switch their position and convey very similar meaning. {\color{myorange}So we believe that {\PUAfont \symbol{"0135}} reads as "Ah",} {\color{myorange}{\PUAfont \symbol{"0119}} is geminate consonant.} These symbols can be used to convey frightened or shocked.\\
{\PUAfont \symbol{"0119}}と{\PUAfont \symbol{"0135}}が位置を入れ替えても似た意味を表すというのは非常に奇妙です。したがって、{\color{myorange}{\PUAfont \symbol{"0135}}は「あ」と読み}、{\color{myorange}{\PUAfont \symbol{"0119}}は促音であると考えられます。}これらの記号は、驚いたりショックを受けたりした気持ちを表すのに使用できます。\\
\zhfont{这两种符号通常出现在爱丽丝震惊/害怕的情况下,因此我认为它大概率有元音“啊”的发音。同时,2种符号可以上下颠倒,同时{\PUAfont \symbol{"0119}}一般仅出现一次,{\PUAfont \symbol{"0135}}可以出现多次,所以我们认为{\PUAfont \symbol{"0119}}是促音,{\PUAfont \symbol{"0135}}则读作“啊”}\\
4. {\PUAfont \symbol{"0118}}{\PUAfont \symbol{"0118}}{\PUAfont \symbol{"0118}}\\
This symbol can appear multiple times to convey a very strong emotion, and the emotion seems to be painful or shocked. {\color{myred}We think it reads as "Waaah…".}\\
\jpfont{このシンボルは複数回出現し、非常に強い感情を表すことがあります。その感情は痛みがあったり、ショックを受けたりしているように見えます。{\color{myred}「わわわ」と読めると考えられます。}}\\
\zhfont{我暂时没有想到这种它应该如何读,因为“啊”已经有了对应符号。因此目前它被用[悲鸣]指代}\\
5. {\PUAfont \symbol{"013A}}{\PUAfont \symbol{"017E}}{\PUAfont \symbol{"017E}}\\
This symbol only appeared ones, where Alice woke up. So we believe it sounds like the sound of waking up, for exaple, {\color{myred}Ugh......}.\\
\jpfont{このシンボルは一回だけ、アリスが目を覚ました時のセリフで出ました。そのため、目を覚ますの音、つまり{\color{myred}「ん......」}と読めると考えられます。}
\zhfont{纯猜测}

\subsection{Pronouns / \jpfont{代名詞}}
\subsubsection{Demonstrative pronouns / \jpfont{指示代名詞}}
1. {\PUAfont \symbol{"0125}}\\
{\color{myorange}That・That one / \jpfont{その・それ}}\\
\zhfont{在翻译开始初期的推断,目前来看比较靠谱}\\
2. {\PUAfont \symbol{"0126}}\\
{\color{myorange}This・This one / \jpfont{この・これ}}\\
\zhfont{在翻译开始初期的推断,目前来看比较靠谱}\\
3. {\PUAfont \symbol{"0132}}\\
{\color{myorange}There / \jpfont{そこ}}\\
\zhfont{从爱丽丝指着终端说的话得出}\\

\subsubsection{Personal pronouns / \jpfont{人称代名詞}}
1. {\PUAfont \symbol{"011D}}\\
{\color{mygreen}Me. / \jpfont{わたし。}}\\
2. {\PUAfont \symbol{"0120}}\\
{\color{mygreen}You. / \jpfont{きみ。}}\\

\subsubsection{Interrogative pronouns / \jpfont{疑問代名詞}}
Interrogative pronouns consist of one {\PUAfont \symbol{"0130}}, and one pronoun.\\
\jpfont{疑問代名詞は、1つの{\PUAfont \symbol{"0130}}と1つの代名詞で構成されます。}\\
1. {\PUAfont \symbol{"0130}}{\PUAfont \symbol{"011C}}\\
{\color{myorange}Who / \jpfont{誰}}\\
\zhfont{通过常在代词中出现的偏旁(类似旗子的符号)推测,这应该是在询问“那边是谁?”}\\
2. {\PUAfont \symbol{"0130}}{\PUAfont \symbol{"0127}}\\
{\color{myred}How / \jpfont{どうやって}}\\
\zhfont{从这个符号唯一出现的一次句子来看,那句话可能是“我怎么会做出这种事?”最为合理,因此该词汇被翻译为how}\\
3. {\PUAfont \symbol{"0130}}{\PUAfont \symbol{"013C}}\\
{\color{myorange}Means what / \jpfont{どういう意味}}\\
\zhfont{这个符号本身类似书本,且出现的两次位置都是爱丽丝与诺艾尔尝试进行交流沟通的位置,因此认为是“表达什么”}\\
4. {\PUAfont \symbol{"0130}}{\PUAfont \symbol{"0128}}\\
{\color{mygreen}Where / \jpfont{どこ}}\\
5. {\PUAfont \symbol{"0130}}{\PUAfont \symbol{"0143}}\\
{\color{myorange}Why / \jpfont{どうして}}\\
\zhfont{大家一致认为这个词出现的位置是在说“为什么不杀我”,因此此处翻译为“为什么”}\\
6. {\PUAfont \symbol{"0130}}{\PUAfont \symbol{"0134}}\\
{\color{myorange}For what / \jpfont{どの原因で}}\\
\zhfont{建立在该符号表示原因/结果的基础上}\\
7. {\PUAfont \symbol{"0130}}\\
{\color{myorange}Pure confuse / \jpfont{単純な疑問}}\\
\zhfont{不排除该标志单独出现的可能性。有可能它与其他词相连但并不组成疑问代词}\\

\subsection{Adverb / \jpfont{副詞}}
1. {\PUAfont \symbol{"011E}}\\ 
{\color{myorange}Please / \jpfont{...してください}}\\
\zhfont{它看起来像是Please的前两个字母发生了一定变化产生。同时从大量语料综合来看,几乎一定在表达请求}\\
2. {\PUAfont \symbol{"0118}}\\
{\color{myorange}Convey want something happen, used after verb. / \jpfont{何かが起きてほしいと伝える, 動詞の後に使用します}}\\
\zhfont{当它出现在动词后面时,我有两种猜想:1种是作为语气词,表示想做某件事的程度,另一种是直接作为副词表示想做某事。总之,当这个符号在动词后面出现时,修饰的动词一般都是爱丽丝在当时需要做/想做的}\\

\subsection{Verb / \jpfont{動詞}}
1. {\PUAfont \symbol{"014B}}\\
{\color{myorange}Feel / \jpfont{感じ}}\\
\zhfont{我们实现知道这个词的读音是“ka”,按照已有规律来推断,这个词的日文对应词汇一定以“Ka”开头。综合考虑所有出现区域,我认为最有可能的是“感觉,感到”。另一个可能翻译是“原谅”,但是第一句应该是请求“请原料我”,“我”却出现在原谅的前方,且没有使用please符号,因此我认为这一可能性不高。}\\
2. {\PUAfont \symbol{"0111}}\\
{\color{myorange}Come / \jpfont{来る}}\\
\zhfont{从多处对话综合推断出的概率最高的结果}\\
3. {\PUAfont \symbol{"010B}}\\
{\color{mygreen}Kill / \jpfont{殺す}}\\
4. {\PUAfont \symbol{"0146}}\\
{\color{myorange}Escape / \jpfont{脱出する}}\\
\zhfont{共出现两次,第一次在“那里”之前,第二次在“我”之前,因此推断为离开}\\
5. {\PUAfont \symbol{"010D}}\\
{\color{myorange}See / \jpfont{みる}}\\
\zhfont{出现位置是爱丽丝指着设备时说的台词。要么是“看”,要么是”去“,讨论后决定翻译为”看“}\\
6. {\PUAfont \symbol{"0106}}\\
{\color{myorange}Believe・Think / \jpfont{思う・信じる}}\\
\zhfont{非常常见,但是不容易推断的一个词。根据碑林的观点,翻译为”认为/相信“}\\
7. {\PUAfont \symbol{"010A}}\\
{\color{mygreen}Battle・Fight / \jpfont{戦う}}\\
8. {\PUAfont \symbol{"0107}}\\
{\color{myorange}Find・Chase / \jpfont{探す・追いかける}}\\
\zhfont{关键位置有2处:一处是爱丽丝在被诺艾尔发现时说“那个精灵...”, 另一个位置是爱丽丝在要求诺艾尔相信她时,说“因为???了那个,所以...”,因此我认为是与“找到”含义接近的词。}\\
9. {\PUAfont \symbol{"0105}}\\
{\color{myorange}Do / \jpfont{する}}\\
\zhfont{从形状就可以相信这个词是动词的基础,因此认为它是“做...”比较合理}\\
10. {\PUAfont \symbol{"0108}}\\
{\color{myred}Reconcile / \jpfont{和解する}}\\
\zhfont{它的形状与“战斗”动词的下半部分正好对应,结合出现位置,有理由相信它的含义与战斗的反面,也就是和解有关。}\\
11. {\PUAfont \symbol{"010C}}\\
{\color{myorange}receive...from...・get... from...・Some do... for me / \jpfont{くれる}}\\
\zhfont{这个词以“ku”作为开头发音,在按照出现频率考虑所有以ku开头发音的日语单词,并尝试带入后,最有可能的结果就是“...为我做”}\\
12. {\PUAfont \symbol{"010E}}\\
{\color{mygreen}have / \jpfont{持つ}}\\
13. {\PUAfont \symbol{"0153}}\\
{\color{myorange}Understand / \jpfont{理解}}\\
\zhfont{这个字读音为ri,出现位置是爱丽丝理解了诺艾尔最后要求她离开的请求,因此大概率为“理解”}\\
14. {\PUAfont \symbol{"0110}}\\
{\color{myorange}Convey / \jpfont{伝える}}\\
\zhfont{从多个出现位置的句子推断得出的一个可能含义是“传达/表达”}\\
15. {\PUAfont \symbol{"0112}}\\
{\color{myred}Escape / \jpfont{脱出する}}\\
\zhfont{爱丽丝在想找小狗时多次说到这个动词}\\
16. {\PUAfont \symbol{"0113}}\\
({\color{myred}[We are unable to translate this]})({\color{myred}\jpfont{[合理的な翻訳は得られない]}})\\
\zhfont{暂时没有可信翻译结果}\\
17. {\PUAfont \symbol{"0115}}\\
{\color{myred}Ask・Question / \jpfont{聞く・疑問する}}\\
\zhfont{纯猜测,因为此处在询问爱丽丝姓名,因此爱丽丝也许会问:“你想问的是什么?”}\\
18. {\PUAfont \symbol{"0116}}\\
({\color{myred}[We are unable to translate this]})({\color{myred}\jpfont{[合理的な翻訳は得られない]}})\\
\zhfont{暂时没有可信翻译结果}\\


\subsection{adjective / \jpfont{形容詞}}
1. {\PUAfont \symbol{"011F}}\\
({\color{myred}[We are unable to translate this]})({\color{myred}\jpfont{[合理的な翻訳は得られない]}})\\
\zhfont{暂时没有可信翻译结果}\\
2. {\PUAfont \symbol{"0122}}\\
{\color{myred}Bad・Evil / \jpfont{悪い・悪っぽい}}\\
\zhfont{仅出现一次,是爱丽丝在信任诺艾尔后说,“她不是...”时使用的形容词。因此认为这是一个负面的词,但其实可能性很多}\\
3. {\PUAfont \symbol{"0129}}\\
{\color{myred}Dangerous / \jpfont{危険}}\\
\zhfont{出现位置是小狗欺负诺艾尔时,爱丽丝说“请不要战斗,...”,因此认为这个词含义是“危险”}\\
4. {\PUAfont \symbol{"0149}}\\
({\color{myred}[We are unable to translate this]})({\color{myred}\jpfont{[合理的な翻訳は得られない]}})\\
\zhfont{暂时没有可信翻译结果 备注:日本读音为i(い)}\\

\subsection{Nouns / \jpfont{名詞}}
\subsubsection{Collective nouns / \jpfont{集合名詞}}
1. {\PUAfont \symbol{"0123}}\\
{\color{myorange} Monster / \jpfont{魔族}}\\
\zhfont{该词汇的右侧偏旁非常接近“魔物”的词根,因此认为这个词是形容魔物的集合名词。然而,这个翻译套入目前的语料后比较奇怪,但我也没有找到更好的翻译。}\\
2. {\PUAfont \symbol{"0121}}\\
{\color{myorange} Elf / \jpfont{エルフ}}\\
\zhfont{与“魔族”对照得出,但是实际上还有其余可能性,比如“那个人”}\\
3. {\PUAfont \symbol{"0136}}\\
{\color{myorange}Friend・Partner / \jpfont{友達・仲間}}\\
\zhfont{通过爱丽丝发现小狗不见了以后说的话得出。考虑到小狗名字里有这个字,这个字表示积极意义的概率很大。因此认为是“朋友”}\\
\subsubsection{Other nouns / \jpfont{他の名詞}}
1. {\PUAfont \symbol{"0100}}\\
{\color{myorange}Future / \jpfont{未来}}\\
\zhfont{通过碑林的推测,以及符号类似"Future"的首个字母推断}\\
2. {\PUAfont \symbol{"012A}}\\
{\color{myorange}Correct / \jpfont{正確}}\\
\zhfont{关键位置:诺艾尔正确发音爱丽丝的名字时,爱丽丝所说的单个字符。}\\
3. {\PUAfont \symbol{"012F}}\\
{\color{myorange}None・Nothing (Bad) happend / \jpfont{なし・無事}}\\
\zhfont{争议词汇}\\
4. {\PUAfont \symbol{"013F}}\\
({\color{myred}[We are unable to translate this]})({\color{myred}\jpfont{[合理的な翻訳は得られない]}})\\
\zhfont{从形状推断}\\
5. {\PUAfont \symbol{"013D}}\\
{\color{myorange}Meaning / \jpfont{意味・内容}}\\
\zhfont{这个词在表示“说/传达”的动词下方,因此有理由相信它的含义是“内容”}\\

\subsection{Copula / \jpfont{コピュラ}}
1. {\PUAfont \symbol{"013B}}\\
{\color{myorange}be・is / \jpfont{…は…}}\\
\zhfont{多个位置综合推断}\\

\subsection{Uncategorized / \jpfont{未分類}}
1. {\PUAfont \symbol{"0137}}\\
{\color{mygreen}Negative marker / \jpfont{否定記号/否定助動詞}}\\
2. {\PUAfont \symbol{"0139}}\\
{\color{mygreen}Past tense marker / \jpfont{過去形マーカー}}\\
3. {\PUAfont \symbol{"012E}}\\
{\color{myred}It is ... / \jpfont{…ことは…}}\\
\zhfont{一共出现2次,一次是“请不要战斗+这个词+形容词”,一次是“这个词+那个”,非常难以翻译,目前暂时认为这是一个以修饰用词,表面...是...的,或者表示“是...”,从而第一句翻译为“请不要战斗,这个行为很...”,第二句翻译为“是那样啊”}\\
4. {\PUAfont \symbol{"011F}}{\PUAfont \symbol{"0100}}{\PUAfont \symbol{"012D}}\\
{\color{myred}As soon as possible / \jpfont{早く}}\\
\zhfont{非常难搞的一个词,目前第三个元素只在这个词出现,第二个词认为表示“未来”,综合这个词汇出现的位置,认为它表示“尽可能快地...”/“快做...”}\\
5. {\PUAfont \symbol{"011A}}\\
({\color{myred}[We are unable to translate this]})({\color{myred}\jpfont{[合理的な翻訳は得られない]}})\\
\zhfont{暂时没有可信翻译结果}\\
6. {\PUAfont \symbol{"012B}}\\
({\color{myred}[We are unable to translate this]})({\color{myred}\jpfont{[合理的な翻訳は得られない]}})\\
\zhfont{暂时没有可信翻译结果}\\
7. {\PUAfont \symbol{"012C}}\\
({\color{myred}[We are unable to translate this]})({\color{myred}\jpfont{[合理的な翻訳は得られない]}})\\
\zhfont{暂时没有可信翻译结果}\\
8. {\PUAfont \symbol{"0131}}\\
{\color{myred}Dilemma / \jpfont{ジレンマ・窮地}}\\
\zhfont{从爱丽丝被逼入绝境,进退两难的情况下说出的句子推断得出}\\
9. {\PUAfont \symbol{"0133}}\\
{\color{myorange}So・Then / \jpfont{だから・なら}}\\
\zhfont{早期从形状得出的推断,感觉没毛病}\\
10. {\PUAfont \symbol{"0134}}\\
{\color{myorange}Because / \jpfont{…ので・…の原因で}}\\
\zhfont{早期从形状得出的推断,感觉没毛病}\\
11. {\PUAfont \symbol{"013E}}\\
{\color{myred}For... / \jpfont{...の目的・ためで}}\\
\zhfont{从形状和出现句子上下文推断}\\
12. {\PUAfont \symbol{"0142}}\\
{\color{myred}Up・Before... / \jpfont{上・...の前に}}\\
13. {\PUAfont \symbol{"0154}}\\
{\color{myred}Down・After / \jpfont{下・...の後に}}\\
\zhfont{暂时没有可信翻译结果}\\
14. {\PUAfont \symbol{"0144}}\\
({\color{myred}[We are unable to translate this]})({\color{myred}\jpfont{[合理的な翻訳は得られない]}})\\
\zhfont{暂时没有可信翻译结果}\\
15. {\PUAfont \symbol{"0145}}\\
{\color{myred}Already / \jpfont{も... ・もはや...}}\\
\zhfont{从遇到堵死的路的位置猜测得出}\\

\newpage
%%%%%%%%%%%%%%%%%%%%%%%%%%%%%%%%%%%%%%%%%%%%%%%%%%%%%%%%%%%%%%%
% In this section, you should write:
% 1. The screenshot of each sentence appeared in the live
% 2. Analyze and translation of them
%%%%%%%%%%%%%%%%%%%%%%%%%%%%%%%%%%%%%%%%%%%%%%%%%%%%%%%%%%%%%%%
\section{Translation of sentences / \normalfont{\jpfont{セリフの翻訳}}}
We are aiming at translating all dialog, so we have to speculate, even guess some results. This is because we want the reader to have a smooth reading experience. We apologize for all potential mistakes.\\\\
\jpfont{すべてのセリフを翻訳することを目指しているため、推測や推測を含む場合もあります。これは、読者の皆様にスムーズな体験を提供したいためです。誤りがあった場合はお詫び申し上げます。}
\subsection{Dialogue appeared in stream / \normalfont{\jpfont{配信であるセリフ}}}
  \textbf{Conversation 1}\\
    Noel: ...\\
    Noel: \jpfont{見つけた!}\\
    \zhfont{……!找到你了!}\\
    Alice: {\PUAfont \symbol{"0121}}{\PUAfont \symbol{"0111}}{\PUAfont \symbol{"0139}}{\PUAfont \symbol{"017F}}({\color{myorange}The elf has come!})({\color{myorange}\jpfont{エルフがきた!}})\\\\
  \textbf{Conversation 2}\\
    (Noel saw alice, and alice fell on the ground)\\
    (\jpfont{アリスはノエルに見つけられて, 地面に転んだ。})\\
    Alice: {\PUAfont \symbol{"0135}}{\PUAfont \symbol{"0119}}{\PUAfont \symbol{"017E}}\hspace{0.5cm}{\PUAfont \symbol{"0135}}{\PUAfont \symbol{"017E}}{\PUAfont \symbol{"017E}}({\color{myorange}Ah... ...})(\jpfont{{\color{myorange}アッ… ア…}})\\
    Alice: {\PUAfont \symbol{"011E}}{\PUAfont \symbol{"0137}}{\PUAfont \symbol{"0111}}{\PUAfont \symbol{"017F}}({\color{myorange}Do not come any closer!})({\color{myorange}\jpfont{来ないで!}})\\\\
  \textbf{Conversation 3}\\
    (Noel saw alice)\\
    (\jpfont{アリスはノエルに見つけられた。})\\
    Alice: {\PUAfont \symbol{"0118}}{\PUAfont \symbol{"0118}}{\PUAfont \symbol{"0118}}{\PUAfont \symbol{"0118}}{\PUAfont \symbol{"017F}}({\color{myred}Waaah…})({\color{myred}\jpfont{わわわわ!}})\\
    Alice: {\PUAfont \symbol{"0135}}{\PUAfont \symbol{"0119}}\hspace{0.5cm}{\PUAfont \symbol{"0137}}\hspace{0.5cm}{\PUAfont \symbol{"011D}}{\PUAfont \symbol{"014B}}{\PUAfont \symbol{"0118}}\hspace{0.5cm}{\PUAfont \symbol{"017E}}{\PUAfont \symbol{"011E}}{\PUAfont \symbol{"0137}}{\PUAfont \symbol{"0111}}{\PUAfont \symbol{"017E}}{\PUAfont \symbol{"017F}}({\color{myorange}Ah, No}, {\color{myred}I feel scared} ...{\color{myorange}Do not come any closer...!})({\color{myorange}\jpfont{アッ、やだ}{\color{myred}、怖い、}{\color{myorange}来ないで!}})\\
    (Alice's pet attacked noel and fell on ground)\\
    Alice: {\PUAfont \symbol{"0101}}{\PUAfont \symbol{"0148}}{\PUAfont \symbol{"0147}}{\PUAfont \symbol{"017F}}({\color{mygreen}[Pet's Name]!})({\color{mygreen}\jpfont{[ペットの名前]!]}})\\
    Alice: {\PUAfont \symbol{"011F}}{\PUAfont \symbol{"0100}}{\PUAfont \symbol{"012D}}{\PUAfont \symbol{"011E}}{\PUAfont \symbol{"0146}}{\PUAfont \symbol{"0126}}{\PUAfont \symbol{"0128}}{\PUAfont \symbol{"017F}}({\color{myred}Escape from }{\color{myorange}there} {\color{myred}as soon as possible!})({\color{myred}\jpfont{早く}{\color{myorange}そこ}{\color{myred}から脱出してください!}})\\\\
  \textbf{Conversation 4}\\
    Noel: \jpfont{ちょっと...}\\
    Noel: \jpfont{いい加減、捕まって!}\\
    \zhfont{差不多得了……快点乖乖投降吧!\\}
    Alice: {\PUAfont \symbol{"0135}}{\PUAfont \symbol{"0119}}{\PUAfont \symbol{"017E}}\hspace{0.5cm}{\PUAfont \symbol{"014A}}{\PUAfont \symbol{"0121}}{\PUAfont \symbol{"0107}}{\PUAfont \symbol{"011A}}{\PUAfont \symbol{"017F}}({\color{myorange}Ah... That elf} {\color{myred}persists in chasing me!})(\jpfont{{\color{myorange}アッ... そのエルフは}{\color{myred}追いかけ続ける!}})\\
    Alice: {\PUAfont \symbol{"0106}}\hspace{0.5cm}{\PUAfont \symbol{"011D}}{\PUAfont \symbol{"0106}}{\PUAfont \symbol{"010C}}{\PUAfont \symbol{"011F}}{\PUAfont \symbol{"0100}}{\PUAfont \symbol{"012D}}{\PUAfont \symbol{"017F}}({\color{myred}Think, Think an idea} {\color{myorange}for }{\color{mygreen}me} {\color{myred}quickly!})(\jpfont{{\color{myred}考えて、}{\color{mygreen}私}{\color{myred}に早く何かを考えてくれて!}})\\\\
  \textbf{Conversation 5}\\
    (Alice was seen by Noel)\\
    (\jpfont{アリスはノエルに見つけられた。})\\
    Alice: {\PUAfont \symbol{"0119}}{\PUAfont \symbol{"0135}}{\PUAfont \symbol{"017E}}({\color{myorange}Ah...})({\color{myorange}\jpfont{ッア...}})\\
    Alice: {\PUAfont \symbol{"0145}}{\PUAfont \symbol{"017E}}{\PUAfont \symbol{"012F}}{\PUAfont \symbol{"017E}}({\color{myorange}It's already...deadend...})({\color{myorange}\jpfont{もう...ない…}})\\
    (A short pause)\\
    Alice: {\PUAfont \symbol{"0130}}{\PUAfont \symbol{"011C}}{\PUAfont \symbol{"017E}}({\color{myorange}Who's are you…})({\color{myorange}\jpfont{あなたは誰...}})\\
    Alice: {\PUAfont \symbol{"011E}}{\PUAfont \symbol{"0108}}{\PUAfont \symbol{"011D}}{\PUAfont \symbol{"017E}}{\PUAfont \symbol{"017E}}\hspace{0.5cm}{\PUAfont \symbol{"011D}}{\PUAfont \symbol{"0131}}{\PUAfont \symbol{"0133}}{\PUAfont \symbol{"010B}}{\PUAfont \symbol{"0126}}{\PUAfont \symbol{"0128}}{\PUAfont \symbol{"017E}}{\PUAfont \symbol{"017E}}({\color{myorange}Please }{\color{myred}reconcile with} {\color{mygreen}me...... }{\color{myred}If I don't have choice, I will }{\color{mygreen}kill} {\color{myred}you} {\color{mygreen}here......})(\jpfont{{\color{mygreen}私}{\color{myred}と和解して……他に選択肢がないなら、{\color{mygreen}ここで殺し}{\color{myred}てやる}})\\
    Noel: \jpfont{行き止まり、だね...}\\
    Noel: \jpfont{もう、逃がさない!}\\
    \zhfont{走投无路了吧......你已经逃不了了!}\\
    Noel: \jpfont{大丈夫、大人しくしてくれれば何も痛いことはしません。}\\
    Noel: \jpfont{投降して、ね?}\\
    \zhfont{没关系的。只要你乖乖配合,我是不会伤害你的。放下武器,好吗?}\\
    Alice: {\PUAfont \symbol{"017E}}{\PUAfont \symbol{"017E}}{\PUAfont \symbol{"017E}}{\PUAfont \symbol{"017E}}{\PUAfont \symbol{"017E}}{\PUAfont \symbol{"017E}}{\PUAfont \symbol{"017E}}\\
    Alice: {\PUAfont \symbol{"0101}}{\PUAfont \symbol{"0148}}{\PUAfont \symbol{"0147}}{\PUAfont \symbol{"017F}}{\PUAfont \symbol{"017F}}({\color{mygreen}[Pet's name]!!})({\color{mygreen}\jpfont{[ペットの名前]!!}})\\
    (Noel was attacked by {\PUAfont \symbol{"0101}}{\PUAfont \symbol{"0105}}{\PUAfont \symbol{"0147}})\\
    ({\PUAfont \symbol{"0101}}{\PUAfont \symbol{"0105}}{\PUAfont \symbol{"0147}}\jpfont{はノエルに攻撃しました。})\\
    Noel: \jpfont{ぐ......っ!}\\
    Noel: \jpfont{かはっ、苦しい......!}\\
    \zhfont{呃......!喘、喘不过气......!}\\
    Alice: {\PUAfont \symbol{"0135}}{\PUAfont \symbol{"017E}}\hspace{0.5cm}{\PUAfont \symbol{"011E}}{\PUAfont \symbol{"0109}}{\PUAfont \symbol{"017F}}({\color{myorange}Ah...}{\color{mygreen}Please bear with it!})(\jpfont{{\color{myorange}ア…}{\color{mygreen}我慢して!}})\\
    Alice: {\PUAfont \symbol{"011E}}{\PUAfont \symbol{"0137}}{\PUAfont \symbol{"010A}}{\PUAfont \symbol{"012E}}{\PUAfont \symbol{"0129}}{\PUAfont \symbol{"017E}}{\PUAfont \symbol{"017F}}({\color{mygreen}Please do not fight}{\color{myred}, it is dangerous!})(\jpfont{{\color{mygreen}戦わないで}{\color{myred}、それは危険です…!}})\\
    Alice: {\PUAfont \symbol{"0103}}{\PUAfont \symbol{"0104}}{\PUAfont \symbol{"0120}}{\PUAfont \symbol{"010C}}{\PUAfont \symbol{"014A}}{\PUAfont \symbol{"0121}}{\PUAfont \symbol{"017E}}({\color{myred}Perhaps, }{\color{myorange}that Elf }{\color{myred}want to do protect me from }{\color{myorange}you...?})({\color{myred}\jpfont{もしかして、}{\color{myorange}そのエルフがあなたから}{\color{myred}私を守りたい…?}})\\
    Alice: {\PUAfont \symbol{"0102}}{\PUAfont \symbol{"0130}}{\PUAfont \symbol{"0127}}{\PUAfont \symbol{"0106}}{\PUAfont \symbol{"0105}}{\PUAfont \symbol{"011D}}({\color{myred}How could }{\color{mygreen}I }{\color{myorange}want to do }{\color{myred}such a thing?})(\jpfont{{\color{myred}なんで}{\color{mygreen}私}{\color{myred}があんなことを}{\color{myorange}しようとしたんだ?}})\\
    Noel: \jpfont{油断した......}\\
    Noel: \jpfont{こいつ、ただの子犬型じゃない......?}\\
    \zhfont{大意了......这家伙,不只是普通的幼犬型魔族......?}\\
    Noel: \jpfont{くう、いたた......}\\
    Noel: \jpfont{抜け出すのはムリみたい。}\\
    Noel: \jpfont{なら......止むを得ない......!}\\
    \zhfont{呜、好痛......看来挣脱不掉了。那就......只能这样了......!}\\
    (Noel used holy brust, then the ground collapsed)\\
    (\jpfont{ノエルはホーリーバーストを発動して、地面が崩れた。})\\
    Noel: \jpfont{しまった......!}\\
    \zhfont{糟了......!}\\
    Alice: {\PUAfont \symbol{"0119}}{\PUAfont \symbol{"0135}}{\PUAfont \symbol{"0135}}{\PUAfont \symbol{"0135}}{\PUAfont \symbol{"0135}}{\PUAfont \symbol{"0135}}{\PUAfont \symbol{"017F}}({\color{myorange}Ahhhhhhh!})({\color{myorange}\jpfont{ッアアアアア!}})\\\\
\textbf{Conversation 6}\\
    Noel: \jpfont{っ、}\\
    Noel: \jpfont{いたたた......}\\
    \zhfont{呜!痛痛痛......}\\
    Noel: \jpfont{っ杖!杖は!?}\\
    Noel: \jpfont{来て!}\\
    \zhfont{法杖!我的法杖在哪里!?法杖快飞回来——!}\\
    (Then wand went back to Noel)\\
    (杖はノエルに帰りました。)\\
    Noel: \jpfont{ふぅ......}\\
    \zhfont{呼——}\\
    Noel: \jpfont{体は......何とか大丈夫。}\\
    Noel: \jpfont{でも、結構の高さを落ちちゃったみたい......。}\\
    \zhfont{还好......身体没怎么受伤。不过,这个洞确实好深啊,一不留神就从那么高的地方摔下来了......}\\
    Noel: \jpfont{!魔力植物がたくさん......!}\\
    Noel: \jpfont{この崩落が魔族がやってくるかもしれない......大声も出さない方が良さそう......}\\
    \zhfont{这个洞穴,或许是魔族凿出来的......得小心不能发出太大的声音,把魔族引出来了就糟了......}\\
    (Noel found alice)\\
    (\jpfont{ノエルはアリスを見つけた。})\\
    Alice: {\PUAfont \symbol{"0118}}{\PUAfont \symbol{"0118}}{\PUAfont \symbol{"0118}}{\PUAfont \symbol{"017E}}{\PUAfont \symbol{"017E}}({\color{myred}Waaah…})({\color{myred}\jpfont{わわわ}})\\
    (Alice woke up)\\
    (\jpfont{アリスは目を覚ました})\\
    Alice: {\PUAfont \symbol{"013A}}{\PUAfont \symbol{"017E}}{\PUAfont \symbol{"017E}}\hspace{0.5cm}{\PUAfont \symbol{"0119}}{\PUAfont \symbol{"0135}}{\PUAfont \symbol{"0135}}{\PUAfont \symbol{"017E}}{\PUAfont \symbol{"017E}}({\color{myred}Ugh......} {\color{myorange}Ahhhh......})(\jpfont{{\color{myred}うーん……{\color{myorange}ッアア……}})\\
    Noel: \jpfont{あ、あなた......大丈夫?}\\
    Noel: \jpfont{......じゃなくて。}\\
    Noel: \jpfont{ようやく追い詰めました。}\\
    Noel: \jpfont{......端末は?}\\
    Noel: \jpfont{どこに隠したの?}\\
    \zhfont{你、你没事吧?不、不对,这不是重点......你现在无路可逃了吧,所以,终端呢?你把它藏在哪里了?}\\
    Alice: {\PUAfont \symbol{"0119}}{\PUAfont \symbol{"0135}}{\PUAfont \symbol{"0135}}{\PUAfont \symbol{"017E}}{\PUAfont \symbol{"017E}}{\PUAfont \symbol{"017F}}({\color{myorange}Ahh......!})({\color{myorange}\jpfont{ッアア......!}})\\
    Alice: {\PUAfont \symbol{"0102}}{\PUAfont \symbol{"0130}}({\color{myorange}What?})({\color{myorange}\jpfont{え?}})\\
    Alice: {\PUAfont \symbol{"0102}}\hspace{0.5cm}\jpfont{タンマツ......}({\color{mygreen}Terminal......?})({\color{mygreen}\jpfont{たんまつ......?}})\\
    Alice: {\PUAfont \symbol{"0137}}{\PUAfont \symbol{"017F}}({\color{myorange}No!})({\color{myorange}\jpfont{やだ!}})\\
    Alice: {\PUAfont \symbol{"0137}}\hspace{0.5cm}{\PUAfont \symbol{"011E}}{\PUAfont \symbol{"0137}}{\PUAfont \symbol{"0111}}{\PUAfont \symbol{"017F}}{\PUAfont \symbol{"017F}}({\color{myorange}No, Do not come any closer!})({\color{myorange}\jpfont{やだ、来ないで!}})\\
    Noel: \jpfont{もう逃げたりしないでね?}\\
    Noel: \jpfont{端末はどこ?}\\
    Noel: \jpfont{私はそれを探しに来たの!}\\
    \zhfont{别再逃了,好吗?终端在哪里?我就是为了找那个才来的!}\\
    Alice: {\PUAfont \symbol{"0137}}{\PUAfont \symbol{"017E}}{\PUAfont \symbol{"017E}}{\PUAfont \symbol{"017F}}({\color{myorange}No......!})({\color{myorange}\jpfont{いえ......!}})\\
    (Alice pointed at the back of Noel)\\
    (\jpfont{アリスはノエルの後ろに指さした})\\
    Alice: {\PUAfont \symbol{"0135}}\hspace{0.5cm}{\PUAfont \symbol{"0132}}\hspace{0.5cm}{\PUAfont \symbol{"011E}}{\PUAfont \symbol{"010D}}{\PUAfont \symbol{"0132}}{\PUAfont \symbol{"017F}}({\color{myorange}Ah, There, Please look there!})({\color{myorange}\jpfont{ア、そこ、あそこを見て!}})\\
    Noel: \jpfont{え、何?}\\
    Noel: \jpfont{後ろを指さして......}\\
    Noel: \jpfont{......はっ。}\\
    \zhfont{诶,后面?你是指后面有什么东西吗............啊——}\\
    Noel(Think): \jpfont{この子、魔族を手なづけてるんだった。}\\
    Noel(Think): \jpfont{目を離さないようにしないと......。}\\
    \zhfont{对了,我记得......这个女孩子驯服了一只魔族的样子,得小心,不能把视线移开......}\\
    Noel(Think): \jpfont{あぶないあぶない。}\\
    Noel(Think): \jpfont{その手乗らないからね!}\\
    \zhfont{好险好险......我可不会上你的当!}\\
    Alice: {\PUAfont \symbol{"011D}}{\PUAfont \symbol{"0139}}{\PUAfont \symbol{"0107}}{\PUAfont \symbol{"0125}}{\PUAfont \symbol{"017F}}({\color{myorange}I have found that!})({\color{myorange}\jpfont{それはもう見つけた!}})\\
    Alice: {\PUAfont \symbol{"011D}}{\PUAfont \symbol{"0139}}{\PUAfont \symbol{"0110}}{\PUAfont \symbol{"0134}}{\PUAfont \symbol{"011D}}{\PUAfont \symbol{"0139}}{\PUAfont \symbol{"0107}}{\PUAfont \symbol{"017F}}({\color{myorange}I spoke to you because I have found that!})({\color{myorange}\jpfont{それはもう見つけ、そのため君と話した!}})\\
    (Noel went back to alice)\\
    Alice: {\PUAfont \symbol{"0132}}\hspace{0.5cm}{\PUAfont \symbol{"011E}}{\PUAfont \symbol{"010D}}{\PUAfont \symbol{"0132}}{\PUAfont \symbol{"017F}}({\color{myorange}There, Please look there!})({\color{myorange}\jpfont{あそこ、あそこを見て!}})\\
    Alice: {\PUAfont \symbol{"0132}}\hspace{0.5cm}{\PUAfont \symbol{"011E}}{\PUAfont \symbol{"010D}}{\PUAfont \symbol{"0132}}{\PUAfont \symbol{"017F}}({\color{myorange}There, Please look there!})({\color{myorange}\jpfont{あそこ、あそこを見て!}})\\
    Noel: \jpfont{......?}\\
    (Noel went back and found the installer(\jpfont{端末}))\\
    (\jpfont{ノエルは後ろに行って、端末を見つかった。})\\
    Noel: \jpfont{......本当に落ちてた。}\\
    Noel: \jpfont{これで、盗まれた端末は無事に回収......と}\\
    \zhfont{...还真是掉在这里了。算了,至少被盗走的终端算是顺利找回来了......}\\
    Noel: \jpfont{確かにこの登録端末、セキュリティが設定されてなくて誰でも登録できる状態なんだよね。間違って起動しないようにしないと......。}\\
    \zhfont{这个注册终端,好像没有开启任何安全认证功能,任何人都能随便完成注册呢......得保管好才行,可不能让它意外启动了...}\\
    (Alice looked up, seems to be worried)\\
    (\jpfont{アリスは上を向いて、悲しいみたい。})\\
    Alice: {\PUAfont \symbol{"0135}}{\PUAfont \symbol{"017E}}{\PUAfont \symbol{"017E}}{\PUAfont \symbol{"0119}}{\PUAfont \symbol{"0135}}{\PUAfont \symbol{"017E}}{\PUAfont \symbol{"017E}}({\color{myorange}Ah......Ah......})({\color{myorange}\jpfont{ア......ッア......}})\\
    Alice: {\PUAfont \symbol{"0102}}{\PUAfont \symbol{"0130}}{\PUAfont \symbol{"0128}}{\PUAfont \symbol{"013B}}{\PUAfont \symbol{"0101}}{\PUAfont \symbol{"0148}}{\PUAfont \symbol{"0147}}({\color{myorange}Where is }{\color{mygreen}[Pet's name]?})(\jpfont{{\color{mygreen}[ペットの名前]}{\color{myorange}はどこ?}})\\
    (Alice started crying)\\
    (\jpfont{アリスは泣きました。})\\
    Alice: {\PUAfont \symbol{"011D}}{\PUAfont \symbol{"0106}}{\PUAfont \symbol{"014A}}{\PUAfont \symbol{"0123}}{\PUAfont \symbol{"017E}}{\PUAfont \symbol{"017E}}({\color{mygreen}I }{\color{myorange}want it...})(\jpfont{{\color{mygreen}私は}{\color{myorange}そのこが欲しい…}})\\
    Noel: \jpfont{何を言っているのかわからない、けど......。}\\
    \zhfont{听不太懂你在说什么,不过......}\\
    Noel: \jpfont{もしかして、ここに落としたのもあなたの罠?}\\
    Noel: \jpfont{わざと床が崩落しそうな場所に私を誘導したの?}\\
    \zhfont{莫非,这个地方会坍塌也是你设下的陷阱?你是在故意把我引到这块快塌陷的地面上的吗?}\\
    Alice: {\PUAfont \symbol{"0102}}{\PUAfont \symbol{"0130}}{\PUAfont \symbol{"013C}}{\PUAfont \symbol{"0110}}{\PUAfont \symbol{"0120}}({\color{myorange}What do }{\color{mygreen}you }{\color{myorange}want to convey?})(\jpfont{{\color{mygreen}あなた}{\color{myorange}は何を伝いたいの?}})\\
    Alice: {\PUAfont \symbol{"012E}}{\PUAfont \symbol{"0125}}{\PUAfont \symbol{"017E}}\hspace{0.5cm}{\PUAfont \symbol{"011D}}{\PUAfont \symbol{"0106}}{\PUAfont \symbol{"0101}}{\PUAfont \symbol{"0148}}{\PUAfont \symbol{"0147}}{\PUAfont \symbol{"0134}}{\PUAfont \symbol{"013B}}{\PUAfont \symbol{"0136}}{\PUAfont \symbol{"017E}}({\color{myred}That's how it is... }{\color{mygreen}I }{\color{myorange}want }{\color{mygreen}[Pet's Name] }{\color{myorange}because we are }{\color{myred}partner...})({\color{myred}\jpfont{その通りだ…仲間}{\color{myorange}だから、}{\color{mygreen}私は[ペットの名前]}{\color{myorange}欲しい…}})\\
    Noel: \jpfont{学園にあれだけの魔族を呼んだのはあなた?}\\
    \zhfont{是你把那么多魔族引到学校那边的吗?}\\
    (Alice cried again)\\
    (\jpfont{アリスは再び泣きました。})\\
    Alice: {\PUAfont \symbol{"0125}}{\PUAfont \symbol{"0123}}{\PUAfont \symbol{"0137}}{\PUAfont \symbol{"013B}}{\PUAfont \symbol{"0136}}{\PUAfont \symbol{"017E}}({\color{myorange}It is not }{\color{myred}partner any more...[Because Alice got separated from it]})(\jpfont{{\color{myorange}その子}{\color{myred}はもう仲間で{\color{myorange}はない…}{\color{myred}「アリスがペットと離れてしまったから」}})\\
    Alice: {\PUAfont \symbol{"0135}}{\PUAfont \symbol{"017E}}\hspace{0.5cm}{\PUAfont \symbol{"011D}}{\PUAfont \symbol{"0106}}{\PUAfont \symbol{"0107}}{\PUAfont \symbol{"0125}}{\PUAfont \symbol{"0123}}{\PUAfont \symbol{"017E}}({\color{myorange}Ahh...}\color{mygreen}I }{\color{myorange}want to find it...})(\jpfont{{\color{myorange}ア…}{\color{mygreen}私は}{\color{myorange}その子を探したい…}})\\
    Noel: \jpfont{そうなに慌ててるのはどうして?またわたしを油断させようとしてるのかな。}\\
    (Alice screamed)\\
    (\jpfont{アリスの悲鳴})\\
    Alice: {\PUAfont \symbol{"0137}}\hspace{0.5cm}{\PUAfont \symbol{"0130}}{\PUAfont \symbol{"0134}}{\PUAfont \symbol{"017E}}{\PUAfont \symbol{"017E}}\hspace{0.5cm}{\PUAfont \symbol{"0118}}{\PUAfont \symbol{"0118}}{\PUAfont \symbol{"017F}}\hspace{0.5cm}{\PUAfont \symbol{"011D}}{\PUAfont \symbol{"0112}}{\PUAfont \symbol{"0118}}{\PUAfont \symbol{"017F}}({\color{myorange}No, why...... }{\color{myred}Waaah! }{\color{myorange}Leave }{\color{mygreen}me }{\color{myorange}alone!})(\jpfont{{\color{myorange}いや、どうして……}{\color{myred}わわ!私を離して!}})\\
    Alice: {\PUAfont \symbol{"011D}}{\PUAfont \symbol{"0112}}{\PUAfont \symbol{"0118}}{\PUAfont \symbol{"011D}}{\PUAfont \symbol{"0112}}{\PUAfont \symbol{"0118}}{\PUAfont \symbol{"0134}}{\PUAfont \symbol{"0120}}{\PUAfont \symbol{"017F}}({\color{myorange}Leave me alone! Leave me alone! }{\color{myred}It was all your fault!})({\color{myorange}\jpfont{離して離して!}{\color{myred}全部あなたのせいで!}})\\
    Noel: \jpfont{ちょっ......}\\
    Noel: \jpfont{大きな声を出さないで}\\
    \zhfont{等、等一下,别那么大声!}\\
    Noel: \jpfont{お話、聞かせてもらうだけ。ちょっとわたしとベルミットまで来てもらうね}\\
    \zhfont{我只是想问你几件事而已,能不能跟我一起去贝尔米特呢?}\\
    Alice: {\PUAfont \symbol{"0119}}{\PUAfont \symbol{"0135}}{\PUAfont \symbol{"017E}}{\PUAfont \symbol{"017E}}{\PUAfont \symbol{"011D}}{\PUAfont \symbol{"0112}}{\PUAfont \symbol{"0118}}{\PUAfont \symbol{"017E}}{\PUAfont \symbol{"017E}}({\color{myorange}Ah......Leave me alone......})({\color{myorange}\jpfont{ッア……離して……}})\\
    Alice: {\PUAfont \symbol{"011D}}{\PUAfont \symbol{"0112}}{\PUAfont \symbol{"0118}}{\PUAfont \symbol{"0118}}{\PUAfont \symbol{"0118}}{\PUAfont \symbol{"017E}}{\PUAfont \symbol{"017E}}({\color{myred}Leave me aloneeeeee......})({\color{myred}\jpfont{私を離してっっっ……}})\\
    (Noel thought about her past)\\
    (\jpfont{ノエルは自分の過去を思い出した。})\\
    Noel: \jpfont{............}\\
    Noel: \jpfont{これじゃ、私の方が......。}\\
    \zhfont{要是伤害她的话......那我不就成了......}\\
    (Alice was caught and taken away by spider silk)\\
    (\jpfont{アリスは帰った魔族に捕らえられた。})\\
    Alice: {\PUAfont \symbol{"0137}}{\PUAfont \symbol{"017F}}({\color{mygreen}No!})(\jpfont{{\color{mygreen}いや!}})\\
    Noel: \jpfont{?!}\\
    Noel: \jpfont{しまった、この洞穴の原生魔族が帰ってきただ......!}\\
    \zhfont{糟了,这个洞穴里住着的魔族回来了......!}\\
    Alice: {\PUAfont \symbol{"011E}}{\PUAfont \symbol{"0137}}{\PUAfont \symbol{"0105}}{\PUAfont \symbol{"017F}}{\PUAfont \symbol{"017F}}({\color{myorange}Please don't!})({\color{myorange}\jpfont{しないで!}})\\
    Alice: {\PUAfont \symbol{"011E}}{\PUAfont \symbol{"0112}}{\PUAfont \symbol{"0146}}{\PUAfont \symbol{"011D}}{\PUAfont \symbol{"017F}}{\PUAfont \symbol{"017F}}({\color{myorange}Please release me!!})({\color{myorange}\jpfont{私を離してください!!}})\\
    (Noel fought with the enemy, and claimed victory)\\
    (\jpfont{ノエルは魔族と戦いて、勝利を得た。})\\\\
    \textbf{Conversation 7}\\
    (Alice fell on the ground, seems to be unconscious)\\
    (\jpfont{アリスは地面に落ちて、意識を失った。})\\
    Alice: {\PUAfont \symbol{"0149}}{\PUAfont \symbol{"0144}}{\PUAfont \symbol{"017E}}{\PUAfont \symbol{"017E}}({\color{myred}[We are unable to translate this]})({\color{myred}\jpfont{[合理的な翻訳は得られない]}})\\
    Alice: {\PUAfont \symbol{"011D}}{\PUAfont \symbol{"012B}}{\PUAfont \symbol{"0137}}{\PUAfont \symbol{"0113}}{\PUAfont \symbol{"012E}}{\PUAfont \symbol{"0129}}{\PUAfont \symbol{"017E}}{\PUAfont \symbol{"017E}}({\color{myred}[We are unable to translate this]})({\color{myred}\jpfont{[合理的な翻訳は得られない]}})\\
    (Alice woke up)\\
    (\jpfont{アリスは目を覚ました})\\
    Alice: {\PUAfont \symbol{"017F}}({\color{mygreen}!})\\
    Noel: \jpfont{大丈夫。}\\
    Noel: \jpfont{あなたを攻撃したりはしないよ。}\\
    \zhfont{......别担心。我不会攻击你的。}\\
    Alice: {\PUAfont \symbol{"0102}}{\PUAfont \symbol{"0120}}{\PUAfont \symbol{"0137}}{\PUAfont \symbol{"010B}}{\PUAfont \symbol{"011D}}({\color{mygreen}You }{\color{myorange}are not going to }{\color{mygreen}kill me?})(\jpfont{{\color{mygreen}私を殺}{\color{myorange}さないか?}})\\
    Alice: {\PUAfont \symbol{"0102}}{\PUAfont \symbol{"017E}}{\PUAfont \symbol{"017E}}{\PUAfont \symbol{"0130}}{\PUAfont \symbol{"0143}}({\color{myorange}Why?})({\color{myorange}\jpfont{どうして?}})\\
    Noel: \jpfont{さっき大型との戦闘は......}\\
    Noel: \jpfont{自分の縄張りに侵入した私を排除する目的しか感じなかった}\\
    \zhfont{刚才跟那个大块头的战斗中,我隐约有一种感觉......那家伙是因为我侵入了它的领地才来攻击我的吧。}\\
    Noel: \jpfont{あれはあなたが呼んだ訳じゃないことは、わかるよ}\\
    \zhfont{换句话说,并不是你把它引到这里来的,对吧?}\\
    Noel: \jpfont{だから、一緒にここから出る方法を考えよう}\\
    \zhfont{所以,我们先一起想想办法,看看怎么从这个洞里出去吧。}\\
    Alice: {\PUAfont \symbol{"017E}}{\PUAfont \symbol{"017E}}({\color{mygreen}......})\\\\
    \textbf{Conversation 8}\\
    Noel: \jpfont{ありがとう、助かったよ。}\\
    Noel: \jpfont{改めて挨拶させて。}\\
    Noel: \jpfont{私はノエル。}\\
    \zhfont{谢谢你救了我。重新介绍一下,我是诺艾儿。}\\
    (Alice saw her pet)\\
    (\jpfont{アリスは}{\PUAfont \symbol{"0101}}{\PUAfont \symbol{"0148}}{\PUAfont \symbol{"0147}}を見つけた。)\\
    Alice: {\PUAfont \symbol{"0101}}{\PUAfont \symbol{"0148}}{\PUAfont \symbol{"0147}}{\PUAfont \symbol{"017F}}({\color{mygreen}[Pet's name]!!})({\color{mygreen}\jpfont{[ペットの名前]!!}})\\
    Alice: {\PUAfont \symbol{"0101}}{\PUAfont \symbol{"0148}}{\PUAfont \symbol{"0147}}{\PUAfont \symbol{"013B}}{\PUAfont \symbol{"012F}}({\color{mygreen}[Pet's name]}{\color{myorange} is }{\color{myred}safe and sound.})(\jpfont{{\color{mygreen}[ペットの名前]}{\color{myorange}は}{\color{myred}無事。}})\\
    Alice: {\PUAfont \symbol{"014A}}{\PUAfont \symbol{"0121}}{\PUAfont \symbol{"0137}}{\PUAfont \symbol{"013B}}{\PUAfont \symbol{"0122}}{\PUAfont \symbol{"017E}}{\PUAfont \symbol{"011D}}{\PUAfont \symbol{"014B}}({\color{myorange}That elf is not }{\color{myred}bad}{\color{mygreen}... I }{\color{myorange}feel})(\jpfont{{\color{myorange}そのエルフは}{\color{myred}悪く}{\color{myorange}ない…}{\color{mygreen}私}{\color{myorange}はこう感じます。}})\\
    Noel: \jpfont{......その子、あなたにとってとても大事な存在なんだね}\\
    \zhfont{......那只小家伙,对你来说一定很重要吧。}\\
    Alice: {\PUAfont \symbol{"0102}}{\PUAfont \symbol{"017E}}{\PUAfont \symbol{"017E}}({\color{mygreen}......?})\\
    Noel: \jpfont{わたし、ノエル。あなたは?}\\
    Alice: {\PUAfont \symbol{"0102}}{\PUAfont \symbol{"0126}}{\PUAfont \symbol{"0121}}{\PUAfont \symbol{"0110}}{\PUAfont \symbol{"0115}}{\PUAfont \symbol{"013E}}{\PUAfont \symbol{"013D}}({\color{myorange}Is this Elf }{\color{myred}asking }{\color{myred}for }{\color{myorange}information?})(\jpfont{{\color{myorange}このエルフは}{\color{myred}情報を得るために聞いているの?}})\\
    Alice: \jpfont{......ワタシ......?}\\
    Noel: \jpfont{ノエル。ノ エ ル、だよ}\\
    Alice: \jpfont{...のえ、る?}\\
    Noel: \jpfont{うん。}\\
    Noel: \jpfont{ノエル。}\\
    \zhfont{诺艾儿,诺、艾、儿,记住了吗?}\\
    Alice: {\PUAfont \symbol{"017E}}{\PUAfont \symbol{"017E}}{\PUAfont \symbol{"011B}}{\PUAfont \symbol{"010F}}{\PUAfont \symbol{"0138}}({\color{mygreen}......Alice})({\color{mygreen}\jpfont{......アリス}})\\
    Noel: \jpfont{アリス......}\\
    Noel: \jpfont{あなたの名前、アリスちゃんっていうんだね}\\
    \zhfont{爱丽丝......原来你的名字叫爱丽丝啊。}\\
    Alice: {\PUAfont \symbol{"017E}}{\PUAfont \symbol{"017E}}{\PUAfont \symbol{"012A}}{\PUAfont \symbol{"017F}}({\color{myorange}......Correct!})({\color{myorange}\jpfont{......正確だ!}})\\\\
    \textbf{Conversation 9}\\
    (Noel want to let Alice go before reinforcement come.)\\
    (\jpfont{ノエルは人がくる前に、アリスを帰らせたい。})\\
    Noel: \jpfont{この子は今連れ帰るべきじゃない。}\\
    Noel: \jpfont{きっと、もっと安全で穏やかな出会い方がきっとあるはず。}\\
    \zhfont{......还是让她走吧。总觉得现在不该把她带走呢。未来,肯定会再见面的,更友好、也更和平的见面。}\\
    Alice: {\PUAfont \symbol{"0102}}{\PUAfont \symbol{"017E}}{\PUAfont \symbol{"017E}}({\color{mygreen}......?})\\
    Noel: \jpfont{あ......えっと、人がくるから、ここからすぐに離れて!}\\
    \zhfont{啊......那个,有人来了,快从这里离开!}\\
    Alice: {\PUAfont \symbol{"0102}}{\PUAfont \symbol{"0130}}({\color{myorange}What?})({\color{myorange}\jpfont{え?}})\\
    Noel: \jpfont{つ、通じてない。}\\
    Noel: \jpfont{ジェスチャーで伝えなきゃ}\\
    \zhfont{她、听不懂。必须用手势来传达......}\\
    (Noel used a battle pose to conveny the message)\\
    Alice: {\PUAfont \symbol{"017E}}{\PUAfont \symbol{"017E}}{\PUAfont \symbol{"017E}}({\color{mygreen}.........})\\
    (Alice smiled)\\
    (\jpfont{アリスは笑った。})
    Alice: {\PUAfont \symbol{"012C}}{\PUAfont \symbol{"0117}}{\PUAfont \symbol{"0117}}{\PUAfont \symbol{"0117}}{\PUAfont \symbol{"017F}}({\color{mygreen}UHaHaHaHa!})({\color{mygreen}\jpfont{ウハハハ!}})\\
    Alice: {\PUAfont \symbol{"0102}}{\PUAfont \symbol{"0130}}{\PUAfont \symbol{"013C}}{\PUAfont \symbol{"0110}}{\PUAfont \symbol{"0120}}({\color{myorange}What do }{\color{mygreen}you }{\color{myorange} want to convey?})(\jpfont{{\color{mygreen}君}{\color{myorange}は何を伝いたい?}})\\
    Noel: \jpfont{笑われてる⁉}\\
    \zhfont{被嘲笑了!?}\\
    Noel: \jpfont{人が、くるから、すぐに、逃げて!}\\
    \zhfont{有人,来了,快,逃!}\\
    Alice: {\PUAfont \symbol{"0120}}{\PUAfont \symbol{"010D}}{\PUAfont \symbol{"0106}}{\PUAfont \symbol{"010E}}{\PUAfont \symbol{"013F}}
    ({\color{myred}[We are unable to translate this]})({\color{myred}\jpfont{[合理的な翻訳は得られない]}})\\
    % ({\color{mygreen}You }{\color{myred}are going to show me the black hole spell?})(\jpfont{{\color{mygreen}君}{\color{myorange}はブラックホール魔法を見せてくれる?}})\\
    Alice: {\PUAfont \symbol{"011D}}{\PUAfont \symbol{"0153}}{\PUAfont \symbol{"017E}}{\PUAfont \symbol{"017E}}{\PUAfont \symbol{"0120}}{\PUAfont \symbol{"0137}}{\PUAfont \symbol{"014B}}{\PUAfont \symbol{"0116}}({\color{mygreen}I }{\color{myorange}Understand......}{\color{mygreen}You }{\color{myorange}don't feel }{\color{myred}safe?})(\jpfont{{\color{mygreen}私}{\color{myorange}は理解しました......君は安全だと感じませんか?}})\\
    Alice: {\PUAfont \symbol{"014A}}{\PUAfont \symbol{"0121}}{\PUAfont \symbol{"0110}}{\PUAfont \symbol{"0130}}{\PUAfont \symbol{"0118}}{\PUAfont \symbol{"017E}}{\PUAfont \symbol{"017E}}({\color{myorange}This Elf wants to convey what......})({\color{myorange}\jpfont{このエルフは何かを伝いたい......}})\\
    Alice: {\PUAfont \symbol{"0142}}\hspace{0.5cm}{\PUAfont \symbol{"013D}}{\PUAfont \symbol{"011D}}{\PUAfont \symbol{"0113}}{\PUAfont \symbol{"011F}}{\PUAfont \symbol{"0142}}{\PUAfont \symbol{"0100}}\hspace{0.5cm}{\PUAfont \symbol{"014A}}{\PUAfont \symbol{"0137}}{\PUAfont \symbol{"013B}}{\PUAfont \symbol{"013D}}{\PUAfont \symbol{"0126}}{\PUAfont \symbol{"0121}}({\color{myred}[We are unable to translate first part of the sentence] }{\color{myorange}That is not }{\color{myred}referring to }{\color{myorange}this elf......})(\jpfont{{\color{myred}[文の前半の部分を翻訳できません] }{\color{myred}言っているのは}{\color{myorange}このエルフではない}})\\
    Alice: {\PUAfont \symbol{"0104}} {\PUAfont \symbol{"012A}}{\PUAfont \symbol{"017E}}{\PUAfont \symbol{"017E}}({\color{myred}Isn't that right......?})({\color{myred}\jpfont{違うんじゃない......?}})\\
    Alice: {\PUAfont \symbol{"0102}}{\PUAfont \symbol{"0126}}{\PUAfont \symbol{"0121}}{\PUAfont \symbol{"0110}}{\PUAfont \symbol{"0130}}{\PUAfont \symbol{"011C}}{\PUAfont \symbol{"0121}}{\PUAfont \symbol{"0111}}({\color{myorange}This elf wants to convey that other elf will come?})({\color{myorange}\jpfont{このエルフが伝えたいのは、誰かのエルフもくることか?}})\\
    (Noel received the message that reinforcement would come quickly)\\
    (\jpfont{ノエルは援軍がすぐに来るの消息を受けました。})\\
    Noel: \jpfont{早く!そろそろ他のひとが来るから!}\\
    \zhfont{快点!再过不久,其他人就过来了!}\\
    Alice: {\PUAfont \symbol{"0120}}{\PUAfont \symbol{"0110}}{\PUAfont \symbol{"0139}}{\PUAfont \symbol{"010C}}{\PUAfont \symbol{"011D}}{\PUAfont \symbol{"017E}}{\PUAfont \symbol{"017E}}({\color{myorange}I got what you have said...})({\color{myorange}\jpfont{あなたが話してくれた言葉…}})\\
    Alice: {\PUAfont \symbol{"0106}}{\PUAfont \symbol{"0111}}{\PUAfont \symbol{"0100}}{\PUAfont \symbol{"010D}}{\PUAfont \symbol{"011D}}{\PUAfont \symbol{"0120}}({\color{myorange}I believe we will meet in future......})({\color{myorange}\jpfont{将来私たちが会えることを信じる。}})\\
    Alice: {\PUAfont \symbol{"0119}}{\PUAfont \symbol{"010C}}{\PUAfont \symbol{"010C}}({\color{mygreen}({\color{mygreen}[mild laughter]Heh, heh})\jpfont{っくく}})
\subsection{Appendix / \normalfont{\jpfont{付録}}}
This is a figure posted by AIC develop team in "BiliBili" platform.\\
\jpfont{これはAIC開発チームが「ビリビリ動画」で投稿した図です。}\\
\begin{center}\includegraphics[width=0.8\textwidth]{Figs/Others/shitagi.png}\end{center}
\begin{center}
\begin{tabular}{|*{8}{>{\centering\arraybackslash}p{1.1cm}|}}
\hline
Symbol & {\PUAfont \symbol{"0102}} & {\PUAfont \symbol{"0126}} & {\PUAfont \symbol{"0120}} & {\PUAfont \symbol{"0151}} & {\PUAfont \symbol{"0154}} & {\PUAfont \symbol{"017E}} & {\PUAfont \symbol{"017E}} \\ \hline
Translate & \textcolor{mygreen}{?} & \textcolor{mygreen}{This} & \textcolor{mygreen}{You} & \textcolor{myred}{clothes} & \textcolor{myred}{down} & \textcolor{mygreen}{...} & \textcolor{mygreen}{...} \\ \hline
通訳 & \textcolor{mygreen}{?} & \textcolor{mygreen}{これ} & \textcolor{mygreen}{きみ} & \textcolor{myred}{衣} & \textcolor{myred}{下} & \textcolor{mygreen}{...} & \textcolor{mygreen}{...} \\ \hline
\end{tabular}
\end{center}
We think this is a scene where Noel is teaching Alice her language. And Alice's response is "This is your underwear?".\\
これはノエルがアリスに言語を教えているシーンだと思います。アリスのセリフは”これ、君の下着?”。\\

% \newpage
% \section{Authors / \normalfont{\jpfont{貢献者}}}

% \begin{tabular}{l}
%     \begin{minipage}[c]{0.3\textwidth}
%         \includegraphics[width=3cm, height=3cm, clip, trim=0 0 0 0]{Figs/Authors/test.jpg} \\ % 头像
%     \end{minipage}%
%     \begin{minipage}[c]{0.65\textwidth}
%         \textbf{test1} \\ % 作者名字
%         \small{introduction} \\ % 作者介绍
%         \small{Email: test1@example.com} \\ % 作者邮箱
%     \end{minipage}
% \end{tabular}
% \vspace{0.5cm} \\
% \begin{tabular}{l}
%     \begin{minipage}[c]{0.3\textwidth}
%         \includegraphics[width=3cm, height=3cm, clip, trim=0 0 0 0]{Figs/Authors/test.jpg} \\ % 头像
%     \end{minipage}%
%     \begin{minipage}[c]{0.65\textwidth}
%         \textbf{test2} \\ % 作者名字
%         \small{introduction2} \\ % 作者介绍
%         \small{Email: test2@example.com} \\ % 作者邮箱
%     \end{minipage}
% \end{tabular}



\end{document}
